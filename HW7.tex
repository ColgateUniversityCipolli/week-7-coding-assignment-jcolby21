\documentclass{article}\usepackage[]{graphicx}\usepackage[]{xcolor}
% maxwidth is the original width if it is less than linewidth
% otherwise use linewidth (to make sure the graphics do not exceed the margin)
\makeatletter
\def\maxwidth{ %
  \ifdim\Gin@nat@width>\linewidth
    \linewidth
  \else
    \Gin@nat@width
  \fi
}
\makeatother

\definecolor{fgcolor}{rgb}{0.345, 0.345, 0.345}
\newcommand{\hlnum}[1]{\textcolor[rgb]{0.686,0.059,0.569}{#1}}%
\newcommand{\hlsng}[1]{\textcolor[rgb]{0.192,0.494,0.8}{#1}}%
\newcommand{\hlcom}[1]{\textcolor[rgb]{0.678,0.584,0.686}{\textit{#1}}}%
\newcommand{\hlopt}[1]{\textcolor[rgb]{0,0,0}{#1}}%
\newcommand{\hldef}[1]{\textcolor[rgb]{0.345,0.345,0.345}{#1}}%
\newcommand{\hlkwa}[1]{\textcolor[rgb]{0.161,0.373,0.58}{\textbf{#1}}}%
\newcommand{\hlkwb}[1]{\textcolor[rgb]{0.69,0.353,0.396}{#1}}%
\newcommand{\hlkwc}[1]{\textcolor[rgb]{0.333,0.667,0.333}{#1}}%
\newcommand{\hlkwd}[1]{\textcolor[rgb]{0.737,0.353,0.396}{\textbf{#1}}}%
\let\hlipl\hlkwb

\usepackage{framed}
\makeatletter
\newenvironment{kframe}{%
 \def\at@end@of@kframe{}%
 \ifinner\ifhmode%
  \def\at@end@of@kframe{\end{minipage}}%
  \begin{minipage}{\columnwidth}%
 \fi\fi%
 \def\FrameCommand##1{\hskip\@totalleftmargin \hskip-\fboxsep
 \colorbox{shadecolor}{##1}\hskip-\fboxsep
     % There is no \\@totalrightmargin, so:
     \hskip-\linewidth \hskip-\@totalleftmargin \hskip\columnwidth}%
 \MakeFramed {\advance\hsize-\width
   \@totalleftmargin\z@ \linewidth\hsize
   \@setminipage}}%
 {\par\unskip\endMakeFramed%
 \at@end@of@kframe}
\makeatother

\definecolor{shadecolor}{rgb}{.97, .97, .97}
\definecolor{messagecolor}{rgb}{0, 0, 0}
\definecolor{warningcolor}{rgb}{1, 0, 1}
\definecolor{errorcolor}{rgb}{1, 0, 0}
\newenvironment{knitrout}{}{} % an empty environment to be redefined in TeX

\usepackage{alltt}
\usepackage[margin=1.0in]{geometry} % To set margins
\usepackage{amsmath}  % This allows me to use the align functionality.
                      % If you find yourself trying to replicate
                      % something you found online, ensure you're
                      % loading the necessary packages!
\usepackage{amsfonts} % Math font
\usepackage{fancyvrb}
\usepackage{hyperref} % For including hyperlinks
\usepackage[shortlabels]{enumitem}% For enumerated lists with labels specified
                                  % We had to run tlmgr_install("enumitem") in R
\usepackage{float}    % For telling R where to put a table/figure
\usepackage{natbib}        %For the bibliography
\bibliographystyle{apalike}%For the bibliography
\IfFileExists{upquote.sty}{\usepackage{upquote}}{}
\begin{document}
  \begin{enumerate}
  %%%%%%%%%%%%%%%%%%%%%%%%%%%%%%%%%%%%%%%%%%%%%%%%%%%%%%%%%%%%%%%%%%%%%%%%%%%%%%
  % Question 1
  %%%%%%%%%%%%%%%%%%%%%%%%%%%%%%%%%%%%%%%%%%%%%%%%%%%%%%%%%%%%%%%%%%%%%%%%%%%%%%
    \item Write a \texttt{pois.prob()} function that computes $P(X=x)$, 
    $P(X \neq x)$, $P(X<x)$, $P(X \leq x)$, $P(X > x)$, and $P(X \geq x).$ Enable the user to specify the rate parameter $\lambda$.
\begin{knitrout}\scriptsize
\definecolor{shadecolor}{rgb}{0.969, 0.969, 0.969}\color{fgcolor}\begin{kframe}
\begin{alltt}
\hldef{pois.prob} \hlkwb{<-} \hlkwa{function}\hldef{(}\hlkwc{x}\hldef{,} \hlkwc{lambda}\hldef{,} \hlkwc{type}\hldef{=}\hlsng{"<="}\hldef{) \{}

  \hlcom{# Initialize result variable}
  \hldef{result} \hlkwb{<-} \hlnum{NA}

  \hlcom{# Compute probabilities based on the type argument}
  \hlkwa{if} \hldef{(type} \hlopt{==} \hlsng{"="}\hldef{) \{}
    \hldef{result} \hlkwb{<-} \hlkwd{dpois}\hldef{(x, lambda)}  \hlcom{# P(X = x)}
  \hldef{\}} \hlkwa{else if} \hldef{(type} \hlopt{==} \hlsng{"<"}\hldef{) \{}
    \hldef{result} \hlkwb{<-} \hlkwd{ppois}\hldef{(x} \hlopt{-} \hlnum{1}\hldef{, lambda)}  \hlcom{# P(X < x)}
  \hldef{\}} \hlkwa{else if} \hldef{(type} \hlopt{==} \hlsng{"<="}\hldef{) \{}
    \hldef{result} \hlkwb{<-} \hlkwd{ppois}\hldef{(x, lambda)}  \hlcom{# P(X <= x)}
  \hldef{\}} \hlkwa{else if} \hldef{(type} \hlopt{==} \hlsng{">"}\hldef{) \{}
    \hldef{result} \hlkwb{<-} \hlnum{1} \hlopt{-} \hlkwd{ppois}\hldef{(x, lambda)}  \hlcom{# P(X > x)}
  \hldef{\}} \hlkwa{else if} \hldef{(type} \hlopt{==} \hlsng{">="}\hldef{) \{}
    \hldef{result} \hlkwb{<-} \hlnum{1} \hlopt{-} \hlkwd{ppois}\hldef{(x} \hlopt{-} \hlnum{1}\hldef{, lambda)}  \hlcom{# P(X >= x)}
  \hldef{\}}
  \hlkwd{return}\hldef{(result)}
\hldef{\}}
\end{alltt}
\end{kframe}
\end{knitrout}
  %%%%%%%%%%%%%%%%%%%%%%%%%%%%%%%%%%%%%%%%%%%%%%%%%%%%%%%%%%%%%%%%%%%%%%%%%%%%%%
  % Question 2
  %%%%%%%%%%%%%%%%%%%%%%%%%%%%%%%%%%%%%%%%%%%%%%%%%%%%%%%%%%%%%%%%%%%%%%%%%%%%%%
    \item Write a \texttt{beta.prob()} function that computes $P(X=x)$, 
    $P(X \neq x)$, $P(X<x)$, $P(X \leq x)$, $P(X > x)$, and $P(X \geq x)$
    for a beta distribution. Enable the user to specify the shape parameters
    $\alpha$ and $\beta$.
\begin{knitrout}\scriptsize
\definecolor{shadecolor}{rgb}{0.969, 0.969, 0.969}\color{fgcolor}\begin{kframe}
\begin{alltt}
\hldef{beta.prob} \hlkwb{<-} \hlkwa{function}\hldef{(}\hlkwc{x}\hldef{,} \hlkwc{size}\hldef{,} \hlkwc{prob}\hldef{,} \hlkwc{type}\hldef{=}\hlsng{"<="}\hldef{)\{}
  \hlcom{# Use dbeta and pbeta to conditionally return the correct probability}
\hldef{\}}
\end{alltt}
\end{kframe}
\end{knitrout}
\end{enumerate}
\bibliography{bibliography}
\end{document}
